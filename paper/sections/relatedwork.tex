\section{Related Work}\label{sec:related}
Vast amount of video stored in web archives makes their retrieval based on manual text annotations impractical.
Unsupervised methods provide an edge for videos in terms of action classification, detection and retrieval as manual annotations is not required. 
For retrieval of videos in an unsupervised method we need to exploit the data to find similar semantic and visual properties.

Most of the traditional techniques uses hand crafted features such as Histogram of Oriented Gradients (HOG),Histogram of Optical Flow (HOF), Motion Boundary Histogram
(MBH) around spatio-temporal interest points \cite{Laptev08learningrealistic} or in dense grid to learn features or around dense point trajectories obtained through optical flow based tracking to learn the feature vectors. 

For video retrieval and self supervised learning of the features the direction of time-flow (forward or backward) in videos was studied \cite{Pickupetal14}. CNNs are used to explore the power of learning
slow features, also referred to as “temporally coherent” features\cite{Mobahi2009}
The approach\cite{podlesnaya2016retrieval} does video indexing and retrieval using convulational neural network  based on unified semantic features.

CNN-based unsupervised representation learning method \cite{Misra2016ShuffleAL}where the learning task is to verify whether a sequence
of frames from a video is presented in the correct order or
not.The method does not learn to encode temporal
information but only spatial. In contrast \cite{fernando2017self}
exploits the analogical reasoning over sequences and pose
the feature learning problem as a  $ N + 1 $ way multi class
classification problem which is much harder than the binary
verification problem .

In this paper the methods specified in odd-one-out \cite{fernando2017self} with temporal smoothness is combines to learn features for video retrieval.